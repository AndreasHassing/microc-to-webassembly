\documentclass[a4paper]{article}

\usepackage[english]{babel}
\usepackage[utf8]{inputenc}
\usepackage[hidelinks]{hyperref}
\usepackage{graphicx}
\usepackage{setspace}
\usepackage{pdfpages}
\usepackage{xspace} %Removes space after commands
\usepackage{fancyhdr}
\usepackage{lastpage}
\usepackage{verbatim}
\usepackage{url}
\usepackage{cite}

% Code highlighting
\usepackage{minted}
% Highlight theme
\usemintedstyle{trac}

\pagestyle{fancy}
\fancyhf{}

\lfoot{Page \thepage \hspace{1pt} of \pageref{LastPage}}

\usepackage[margin=0.8in]{geometry}

\title{Bachelor Project: Compiling MicroC to WebAssembly}

\begin{document}
%\pagenumbering{gobble} %Remove page numbers
\begin{titlepage}
	\centering
	{\scshape\LARGE IT University of Copenhagen \par}
	\vspace{1cm}
	{\scshape\Large Bachelor Report\par}
	\vspace{1.5cm}
	{\huge\bfseries Compiling MicroC to WebAssembly \par}
	\vspace{2cm}
	{\Large\itshape Andreas Bjørn Hassing Nielsen\par}
	abhn@itu.dk\\
	\vspace{2cm}
	{\Large Abstract\par}
	%TODO: Write an awesome abstract.
	{\bfseries This is where the abstract will go.}
	\vfill
% Bottom of the page
	{\large \today\par}
\end{titlepage}

\newpage

\tableofcontents
\newpage


\section{Introduction}
The purpose of this project is to build a compiler, also called a translator, from MicroC to WebAssembly, using FsLexYacc\footnote{http://fsprojects.github.io/FsLexYacc/} and F\#\footnote{http://fsharp.org/}.

MicroC is a C-like language described by Peter Sestoft in the book Programming Language Concepts~\cite{PLC}. The original lexer and parser code can be found at his website\footnote{http://www.itu.dk/people/sestoft/plc/}.

%TODO: consider removing this bit about BCD.
Knowledge regarding lexers and parsers in general have been obtained through the book: Basics of Compiler Design~\cite{BCD}, by Torben Mogensen.

\subsection{WebAssembly}
Semantics~\cite{website:wasm-semantics}.

\subsection{Compilers}

\section{Problem Analysis}

\section{User Guide and Examples}

\section{Technical Description}

\section{Testing and Validation}

\section{Peripheral Tooling}

\section{Extensions}

\section{Conclusion}

\section{References}
\begingroup
\renewcommand{\section}[2]{}%
\bibliographystyle{plain}
\bibliography{Bibliography}{}
\endgroup

\section{Appendix}

\end{document}
