\documentclass[a4paper]{article}

\usepackage[english]{babel}
\usepackage[utf8]{inputenc}
\usepackage[hidelinks]{hyperref}
\usepackage{graphicx}
\usepackage{setspace}
\usepackage{pdfpages}
\usepackage{xspace} %Removes space after commands
\usepackage{fancyhdr}
\usepackage{lastpage}
\usepackage{verbatim}
\usepackage{url}

%%% BibTeX %%%
\usepackage{cite}
% JS function to get BibTeX entry from WebAssembly website
\begin{comment}
function BibTeX() {
	let subject = document.title.replace(/ - WebAssembly/, '').replace(' ', '-').toLowerCase();
    return `@misc{website:wasm-${subject}
	author = "The WebAssembly working group",
	title = {{${document.title}}},
	url = "${document.URL}",
	note = {Online; accessed 3 May 2017},
}`;
}
\end{comment}

% source: https://tex.stackexchange.com/a/35044
\makeatletter
\newcommand\footnoteref[1]{\protected@xdef\@thefnmark{\ref{#1}}\@footnotemark}
\makeatother

% Code highlighting
\usepackage{minted}
% Highlight theme
\usemintedstyle{trac}

\pagestyle{fancy}
\fancyhf{}

\lfoot{Page \thepage \hspace{1pt} of \pageref{LastPage}}

\usepackage[margin=0.8in]{geometry}

\title{Bachelor Project: Compiling MicroC to WebAssembly}

\begin{document}
%\pagenumbering{gobble} %Remove page numbers
\begin{titlepage}
	\centering
	{\scshape\LARGE IT University of Copenhagen \par}
	\vspace{1cm}
	{\scshape\Large Bachelor Report\par}
	\vspace{1.5cm}
	{\huge\bfseries Compiling MicroC to WebAssembly \par}
	\vspace{2cm}
	{\Large\itshape Andreas Bjørn Hassing Nielsen\par}
	abhn@itu.dk\\
	\vspace{2cm}
	{\Large Abstract\par}
	%TODO: Write an awesome abstract.
	{\bfseries This is where the abstract will go.}
	\vfill
% Bottom of the page
	{\large \today\par}
\end{titlepage}

\newpage

\tableofcontents
\newpage


\section{Introduction}
The purpose of this project is to build a compiler, also called a translator, from MicroC to WebAssembly, using FsLexYacc\footnote{\label{footnote:fslexyacc-url}http://fsprojects.github.io/FsLexYacc/} and F\#\footnote{http://fsharp.org/}.

MicroC is a C-like language described by Peter Sestoft in the book Programming Language Concepts~\cite{PLC}. The original lexer and parser code can be found at his website\footnote{http://www.itu.dk/people/sestoft/plc/}.

%TODO: consider removing this bit about BCD.
Knowledge regarding lexers and parsers in general have been obtained through the book: Basics of Compiler Design~\cite{BCD}, by Torben Mogensen.

\subsection{WebAssembly}
Semantics~\cite{website:wasm-semantics}.

\subsection{Compilers}

\section{Problem Definition}
WebAssembly (WASM) is a new up-and-coming binary code format for the user-facing web, designed to work alongside JavaScript as a more portable, size- and load-time-efficient, ahead-of-time-compiled alternative. 

The binary WASM format is not bound to be emitted by JavaScript (asm.js) only, potentially bringing your favourite language to the browser front-end. If a language can compile to a intermediate representation supported by WASM (LLVM IR, for instance), it can be compiled to the binary WASM format. 

The purpose of this project is to shed some light on the unfinished WebAssembly specification. To assist in meeting this goal, a compiler will be designed, that compiles from a simple programming language, MicroC, to a WASM format that can be run in a WASM-enabled browser today (currently requiring Firefox Nightly or Chrome Canary with WASM flags set to enabled).

\subsection{Method}
What follows is the planned activities and sources of information.
\begin{itemize}
	\item Translation of MicroC to the binary WebAssembly format, using FsLexYacc\footnoteref{footnote:fslexyacc-url}, as used in the Programs as Data course (fall 2016).
	\item Course books from Programs as Data (fall 2016) will be used as knowledge base for MicroC to WebAssembly translation:
		\begin{itemize}
			\item Peter Sestoft: Programming Language Concepts. Springer 2012.
			\item Torben Mogensen: Basics of Compiler Design. DIKU 2010. Chapters 2 and 3.
		\end{itemize}
	\item Knowledge of WebAssembly will be gained through the official website (http://webassembly.org/) and links branching out from it.
	\item Creation of prototype: web interface that lets users type MicroC in a window, and see it generate WebAssembly in some human-readable format, either s-expression or linear bytecode. The prototype will also give users the ability to run their code as WebAssembly, given that the browser in use supports it.
	\item Reflection over, either the security behind WebAssembly or the portability of it between browsers and platforms.
\end{itemize}

\noindent If time allows:
\begin{itemize}
	\item Extensions to the MicroC language will be implemented.
	\item Simple optimizations to the generated WebAssembly code.
\end{itemize}

\section{Problem Analysis}

\section{User Guide and Examples}

\section{Technical Description}

\section{Testing and Validation}

\section{Peripheral Tooling}

\section{Extensions}

\section{Conclusion}

\section{References}
\begingroup
\renewcommand{\section}[2]{}%
\bibliographystyle{plain}
\bibliography{Bibliography}{}
\endgroup

\section{Appendix}

\end{document}
